
\documentclass[%
 aip,
 jmp,
 amsmath,amssymb,
 preprint,
 reprint,
 author-year,
 author-numerical,
]{revtex4-1}

\usepackage{graphicx}% Include figure files
\usepackage{dcolumn}% Align table columns on decimal point
\usepackage{bm}% bold math
\usepackage{listings}
\usepackage{color}
\usepackage{CJK}
\usepackage[utf8]{inputenc}
%\usepackage[mathlines]{lineno}% Enable numbering of text and display math
%\linenumbers\relax % Commence numbering lines

\lstset{
  frame=tb,
  aboveskip=3mm,
  belowskip=3mm,
  showstringspaces=false,
  columns=flexible,
  basicstyle={\small\ttfamily},
  numbers=none,
  breaklines=true,
  breakatwhitespace=true,
  tabsize=3
}

\begin{document}

\begin{CJK}{UTF8}{gbsn}
%\preprint{AIP/123-QED}

\title[以太经典(ETC)应用市场ICO]{以太经典(ETC)应用市场ICO}% Force line breaks with \\


\author{S.J. Mackenzie}%
 \email{setori88@gmail.com.}

\date{\today}% It is always \today, today,
             %  but any date may be explicitly specified

\begin{abstract}

目前,非技术性终端用户与并不容易明白和使用已发布的加密合约。因此,加密合同不容易达到主流。终端用户都期望可有一个方便的一键式平台,提供用户友好且安全的第三方应用程序,旨在使用加密合约。

\end{abstract}

\keywords{以太经典,flow-based programming ,数据流程,Fractalide,HyperCard}

\maketitle

\begin{quotation}

以太经典(ETC)应用市场ICO旨在提供一个丰富的开发和运行时工具(\textit{Hyperflow}),一个第三方供应商向终端用户销售应用程序的应用市场(\textit{Fractalmarket})以及可将应用程序与etc区块链(\textit{Etherflow})完美集成的一系列组件。。

\end{quotation}

\section{\label{sec:Fractalide}Fractalide}

Fractalide是一个编程语言,它是建基于Flow-based programming,而其中每个组件都是可重新产生,完全可重用的。虽然这个ICO旨在专注于更高层次的项目,如Hyperflow,Etherflow和Fractalmarket,但有一些重要的基础架构工作是需要在Fractalide进行才能实现流畅的前端视觉效果。这样可以清楚地实现硕士论文中详细阐述的card范式。

Fractalide引入的主要概念是可重复使用和可复制的组件。整个平台依赖于这些主要概念。

\subsection{\label{sec:reusability}可重用性}

每个Fractalide组件都可重复使用,因为它具有一个统一的可组合界面。传输到组件中的数据是系统的不变量。

\subsection{\label{sec:reproducibility}再生性}

每个Fractalide组件及其所有依赖性都可以被复制。实际上,组件的整个层次结构可以在其他电脑上再现。每个组件都有一个通用唯一的名称,但是当向程序员呈现组件的名称是可读的。具有相同名称但不同版本的组件可以并排存在于同一个文件系统上,而不会相互矛盾。

\section{\label{sec:Hyperflow}Hyperflow}

Hyperflow是实现HyperCard变体的Fractalide软件模块。 HyperCard的灵感来自Delphi和微软Visual Basic快速应用程序开发平台的,它允许非技术用户快速地组织程序来解决问题。它显著降低了门槛,令几乎没有编程经验的人都能够编程。

HyperCard允许我们将图形用户界面组件拖放到名为卡片的画布上。每个“屏幕”或画布是一张卡片,并且一些卡片形成了一个叠卡片(stack)。比喻说,一张卡是一个网页,一个Stack就是一个网站。 HyperTalk是一种内置编程语言,允许用户写入逻辑令卡片之间移动。

虽然HyperCard 有一些概念是需要保留、删除和改善,以将HyperCard升级成为现代Hypercard,这样就能适合与区块链交互并制作复杂的应用程序。

\subsection{\label{sec:concepts to remove from hypercard}被去除的HyperCard概念}
\subsubsection{\label{sec:HyperTalk}HyperTalk}

HyperTalk是HyperCard的编程语言。这是一个英文的编程语言,下面的一小撮代码就是一个例子。

\begin{lstlisting}
  on mouseUp
    put "100,100" into pos
    repeat with x = 1 to the number of card buttons
      set the location of card button x to pos
      add 15 to item 1 of pos
    end repeat
  end mouseUp
\end{lstlisting}

HyperTalk不是一个特别强大的语言,尽管它确实有狂热的追随者,但不能足以制作可以解决当今昂贵问题的高端应用程式。

最重要的是,HyperTalk不是模块化的,是不能轻易地组合在一起。由于我们对in the small and in the large 在可再利用和可重复的基础设施是有必须的要求。我们将会使用之前提及的组件面向编程语言Fractalide替代HyperTalk。

\subsubsection{\label{sec:HyperCard lacked network access}HyperCard缺乏网路连接}

这是HyperCard死亡的根本原因。当时曾经尝试通过增加网络功能来拯救HyperCard,但是这个任务太难了。史蒂夫·乔布斯(Steve Jobs)最终终止了HyperCard这个项目。如果HyperCard是从全面的网络支持设计出来,HyperCard将能扩展到超越苹果公司(Apple),就像HTTP浏览器变得普遍一样。事实上,我们今天使用的浏览器可能更好。


由于Hyperflow中的每个组件都是从重现性和可重用性的第一个原则构建的,程序员可以轻松地键入组件的名称,并且软件包管理系统和构建系统将立即使该组件在其系统上使用,使打包组件(或组件层次结构并分发同样容易。

\subsection{\label{sec:concepts to keep from hypercard}被保留的Hypercard概念}
\subsubsection{\label{sec:fast switching between run mode and design mode}执行模式和设计模式的快速切换}

HyperCard在运行模式和设计模式之间进行零编译。因此,在设计模式下,可以通过点击运行模式按钮使实验代码运作。在运行模式下,通过点击设计模式按钮,可以再次操作和移动GUI小部件。

这可能是Hyperflow需要复制的其中一个最重要功能,但因为很多原因,这是有难度的。对于文件系统上已经可用的第三方组件层次结构,我们将能够实现这种“零编译”开关,但是在程序员调用组件或整个组件层次结构时,文件系统当前不可用的组件转换使不会顺利。软件包管理系统必须下载所有这些组件的源代码并构建它们,或者在运行模式和设计模式之间的切换期间下载预编译的二进制文件并将其连接到Hyperflow。鉴于我们希望Fractalmarket将承载许多组件层次结构,我们必须接受这些限制。


\section{\label{sec:Fractalmarket}Fractalmarket}

Fractalmarket 希望能提供公平的报酬。这意味着,如果你开发和修复一组特定的组件,当这些组件被使用或在其他应用程序中时,你可以收取费用。当终端用户购买该组件,你作为开发人员将被支付报酬。

因此,如果应用程序开发人员重复使用尽可能多的组件,则最终用户已经购买了许多这些组件的概率较高,因此你的特定层次结构的总价格标签将会减少。这样,您就可以减少下载、编译、安装和付费所需的组件,使你比复制代码和重新发布的程序员更具竞争力。

我们的目标是有很好的指标来证明每个应用层次结构的可重用性因素。Fractalmarket支持一种名为FRACTAL的货币,可以实现买方和卖方之间的交易。在可预见的未来,FRACTAL将是Fractalmarket上唯一被接受的货币。

\section{\label{sec:Etherflow}Etherflow}

Etherflow是一组可重用和可重复的组件,可以实现Hyperflow应用程序和区块链之间的流畅交互。该代码将大量使用ETCDEV团队的软件。

这些组件将允许希望创建区域特定应用程序(如块链保护应用程序)的用户,来呈现拥有所有GUI组件的面向用户应用程序,这样就能透过可重用的区块链组件,容易地使用区块键。
随着用户下载和支付ETCDEV创建的组件,ETCDEV可变得更加可持续,并帮助他们继续致力于以太经典(ETC)社区作出贡献。

\section{\label{sec:conclusion}结论}

我们希望ETC社区会认同「以太经典(ETC) 应用市场ICO」的贡献是有价值的。创建能够除去中间人的全新区块链企业和服务,以及可重用性组件的一个平台,这样就能大大降低成本。

\end{CJK}

\end{document}
